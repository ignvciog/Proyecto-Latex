%Se declara el tipo de documento; podemos indicar el tamaño de la letra, el tamaño del papel. etc
\documentclass[10pt,a4paper]{article}

%Soporte para UTF-8
\usepackage[utf8]{inputenc}

%Se agrega el idioma y los paquetes que extienden la funcionalidad de Latex
\usepackage[spanish, es-noshorthands]{babel}

%Distintos simbolos matemáticos
\usepackage{amsmath,amsfonts,amssymb}

%Utilizando para insertar imágenes
\usepackage{graphicx}

%Paquete para crear hipervínculos
\usepackage[colorlinks,urlcolor=blue]{hyperref}

%Paquete para generar formas
\usepackage{tikz}

%Insertar código coloreado en distintos leanguajes
%\usepackage{minted}

%Indicamos los margenes de nuestro documento
\usepackage[left=2.00 cm, right=2.0cm, top=2cm, bottom=2.00cm]{geometry}

%% Título. Para colocar el título, autor y fecha basta con utilizar.

\title{Nombre del titulo} %Titulo que se vera en la parte superior
\author{Nombre del autor} %Autor que se vera abajo del titulo
\date{} %Fecha

